\documentclass[twocolumn]{article}
\usepackage{graphicx}
\usepackage{amsmath}
\usepackage{amsthm}
 
\theoremstyle{definition}
\newtheorem{definition}{Definition}[section]

% Recap the problem and motivation. This part should be brief, since we've already
% heard about this.

% Briefly describe your overall approach. A small concrete example snippet or two
% would be very helpful.

% Results. Describe what you have achieved. What remains to be done to make this a
% full-fledged research result?

% Lessons Learned. Reflect on the work you did, so you and I can learn from it.
% How did your approach differ from what you originally proposed, and why? What
% was more (or less) challenging than you expected? What would you do differently
% next time? What do you now see as the strengths/weaknesses of a tool like Coq,
% and what does that imply about possible research directions to pursue?

% Code. Please submit all of your Coq code along with some brief documentation so
% I can relatively easily play with it and understand what's going on.
\title{Verified Sentential Decision Diagrams}
\author{Steven Holtzen\\Saketh Ram Kasibatla}
\date{\today}
\begin{document}
\maketitle

\section{Introduction and Motivation}
The goal of this project is to construct and reason about sentential decision
diagrams (SDDs) in Coq. A sentential decision diagram is a data structure which encodes
an arbitrary Boolean formula; see Fig.\ref{fig:sdd} for an example SDD. 

\begin{figure}
  \includegraphics[width=\linewidth]{sdd.png}
  \caption{An SDD for the Boolean formula $f = (A \land B) \lor (B \land C) \lor
    (C \land D)$. Reproduced from \cite{Darwiche2011}.}
  \label{fig:sdd}
\end{figure}

The SDD data structure has several desirable properties:
\begi$$$$
n{enumerate}
\item Canonicity. Any two equivalent Boolean formulas will always compile to
  the same SDD.
\item Determinism. Any given Or-node in an SDD will have at most one true
  child.
\item Decomposability. The children of any given And-node in an SDD share
  no variables.
\end{enumerate} 
These properties allow one to perform many queries in linear time the SDD data
structure, for example: model enumeration, weighted model counting,
satisfiability and validity checking. In our work, we focused on non-canonical
(i.e. uncompressed) SDDs, which are easier to represent and manipulate.

The key operations that we focused on implementing were (1) SDD compilation; 
(2) SDD application; (3) SDD evaluation. 

\subsection{Proof Goals}
Our original goal was to prove the following properties:
  \begin{enumerate}
  \item Correctness: For all Boolean formalae $e$ and inputs $i$,
    \texttt{eval\_boolexpr}($e$, $i$) =
    \texttt{eval\_sdd}(\texttt{compile}($e$),  $i$).
  \item Prove that the resulting SDD is decomposable.
  \item Prove that the resulting SDD is deterministic.
  \item Prove that \texttt{init\_var} generates SDDs with respect to the same
    $v$-tree (will most likely be a necessary lemma for proving the correctness
    of \texttt{apply}).
  \end{enumerate} 
During the course of implementation, these goals proved to be too ambitious. We
were able to prove, subject to certain assumptions, that SDDs remain
deterministic after application. We relaxed goal (1) to be to prove a particular
property of the SDD post-appliction; see Fig.\ref{fig:unsatthm}.

\begin{figure}
\begin{verbatim}
Theorem apply_unsat :
  forall sdd1 sdd2 sddres,
    sdd_unsat sdd1 ->
    sdd_apply OAnd sdd1 sdd2 sddres ->
    sdd_unsat sddres.
\end{verbatim}
  \caption{The apply unsat theorem.}
  \label{fig:unsatthm}
\end{figure}
The proof of this fact required a custom inductive hypothesis on the derivation
of \texttt{sdd\_apply}, which we will elaborate on in later sections. The reason
that this approach can not be extended to prove stronger claims and possbile
future work will also be discussed.

\section{Approach}

\subsection{OCaml Prototype and Fixpoints vs. Inductive Constructions}
We created an initial OCaml implementation of SDD compilation and application to
guide our Coq implementation. This implementation can be found in the
\texttt{ml/} directory of the project.

Some key observations of this OCaml code is that it is reasonably complex; a
direct translation of this code to Gallina was attempted, but ultimately we
decided against it for the following reasons.

\begin{itemize}
\item It was difficult to establish that all the recursive calls (especially
  \texttt{apply}) provably terminated.
\item Gallina did not provide as strong of a structure for proofs (improve this)
\end{itemize} 

\subsection{Inductive SDD Data Structure}
We used the following compact definition for our SDD datastructure:
\begin{verbatim}
Inductive atom : Type :=
| AFalse : atom
| ATrue : atom
| AVar :  nat -> bool -> atom.
Inductive sdd : Type :=
| Or: list (sdd * sdd) -> sdd
| Atom : atom -> sdd.
\end{verbatim}
An \texttt{or} node is simply a list of pairs $(p_i, s_i)$ (primes and subs).
\subsection{SDD Application}
SDD application is the process of combining two SDDs $\alpha$ and $\beta$ into a
third SDD $\gamma$ according to some operation $\circ$. We defined application
as an 4-argument inductive relation:
\begin{verbatim}
Inductive sdd_apply : op -> sdd -> sdd -> sdd -> Prop
\end{verbatim}

where \texttt{op} is either $\land$ or $\lor$. We decomposed this relation in 3
mutually recursive sub-relations:
\begin{enumerate}
\item \texttt{atom\_apply}, which handles applying together atoms
\item \texttt{apply\_or\_list}, two SDD \textt{or}-nodes as arguments and
  produces a new SDD \texttt{or}-node.
\item \texttt{apply\_single\_list}, which takes a prime $p$, a sub $s$, and an
  \texttt{or}-node $[(p_i, s_i)]$ as an argument and produces the SDD of the
  form $[(p_i \land p, s_i \circ s)]$ for all $i$, where $\circ$ is applying
  \texttt{op}.

  For this operation, if $p_i \land p$ is not satisfiable, it is not to be
  included in the final \texttt{or}-node which is produced. This necessitates
  an \texttt{sdd\_sat} and \texttt{sdd\_unsat} relation.
\end{enumerate}

\subsection{Custom Inductive Hypothesis}
Ultimately in order to prove theorems of the form in Fig\ref{fig:unsatthm}, we
need to be able to decompose along the primes and subs of the SDD.


\subsection{SDD V-Tree}\label{sec:vtree}
We experimented with (1) representing the V-Tree as a dependent type of the SDD
and (2) creating a seperate \texttt{sdd\_vtree : sdd -> vtree} inductive relation which
generates a V-Tree from a particular SDD:

\begin{verbatim}
Inductive sdd_vtree : sdd -> vtree -> Prop :=
| AtomTrue : forall n, 
 sdd_vtree (Atom ATrue) (VAtom n)
| AtomFalse : forall n, 
 sdd_vtree (Atom AFalse) (VAtom n)
| AtomVar : forall n b, 
 sdd_vtree (Atom (AVar n b)) (VAtom n)
| OrEmpty : forall v, sdd_vtree (Or []) v
| OrSingle: forall prime sub lvtree rvtree tail, 
 sdd_vtree prime lvtree ->
 sdd_vtree sub rvtree ->
 sdd_vtree (Or (tail)) (VNode lvtree rvtree) ->
 sdd_vtree (Or ((prime, sub) :: tail)) 
  (VNode lvtree rvtree)

\end{verbatim}

\subsection{SDD Compilation}
SDD compilation is the process of transforming a Boolean expression into an SDD.
This procedure requires the following components:
\begin{enumerate}
\item An implementation of Boolean expressions.
\item A method which generates an 
\end{enumerate}

\subsection{Fixed-Width SDD}
\subsection{Custom Inductive Hypotheses}



\bibliographystyle{plain}
\bibliography{bib}
\end{document}
